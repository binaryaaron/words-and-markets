%%%%%latex preamble%%%%%
\documentclass[titlepage]{article}\usepackage[]{graphicx}\usepackage[]{color}
%% maxwidth is the original width if it is less than linewidth
%% otherwise use linewidth (to make sure the graphics do not exceed the margin)
\makeatletter
\def\maxwidth{ %
  \ifdim\Gin@nat@width>\linewidth
  \linewidth
  \else
  \Gin@nat@width
  \fi
}
\makeatother


\usepackage{listings}
\definecolor{mygreen}{rgb}{0,0.6,0}
\definecolor{mygray}{rgb}{0.5,0.5,0.5}
\definecolor{mymauve}{rgb}{0.58,0,0.82}
\lstset{ %
  backgroundcolor=\color{white},   % choose the background color; you must add \usepackage{color} or \usepackage{xcolor}
  basicstyle=\footnotesize,        % the size of the fonts that are used for the code
  breakatwhitespace=false,         % sets if automatic breaks should only happen at whitespace
  breaklines=true,                 % sets automatic line breaking
  captionpos=b,                    % sets the caption-position to bottom
  commentstyle=\color{mygreen},    % comment style
  deletekeywords={...},            % if you want to delete keywords from the given language
  escapeinside={\%*}{*)},          % if you want to add LaTeX within your code
  extendedchars=true,              % lets you use non-ASCII characters; for 8-bits encodings only, does not work with UTF-8
  frame=single,                    % adds a frame around the code
  keepspaces=true,                 % keeps spaces in text, useful for keeping indentation of code (possibly needs columns=flexible)
  keywordstyle=\color{blue},       % keyword style
  language=Python,                 % the language of the code
  morekeywords={*,...},            % if you want to add more keywords to the set
  numbers=left,                    % where to put the line-numbers; possible values are (none, left, right)
  numbersep=5pt,                   % how far the line-numbers are from the code
  numberstyle=\tiny\color{mygray}, % the style that is used for the line-numbers
  rulecolor=\color{black},         % if not set, the frame-color may be changed on line-breaks within not-black text (e.g. comments (green here))
  showspaces=false,                % show spaces everywhere adding particular underscores; it overrides 'showstringspaces'
  showstringspaces=false,          % underline spaces within strings only
  showtabs=false,                  % show tabs within strings adding particular underscores
  stepnumber=2,                    % the step between two line-numbers. If it's 1, each line will be numbered
  stringstyle=\color{mymauve},     % string literal style
  tabsize=2,                       % sets default tabsize to 2 spaces
  title=\lstname                   % show the filename of files included with \lstinputlisting; also try caption instead of title
}
\usepackage{alltt}
\usepackage[sc]{mathpazo}
\usepackage{amsmath, amsthm, amssymb}
\usepackage{graphicx}
\usepackage[T1]{fontenc}
\usepackage{geometry}
\geometry{verbose,tmargin=2.5cm,bmargin=2.5cm,lmargin=1.5cm,rmargin=1.5cm}
%\setcounter{secnumdepth}{2}
%\setcounter{tocdepth}{2}
\usepackage{url}
\usepackage[unicode=true,pdfusetitle,
  bookmarks=true,bookmarksnumbered=true,bookmarksopen=true,bookmarksopenlevel=2,
breaklinks=false,pdfborder={0 0 1},backref=false,colorlinks=false]
{hyperref}
\hypersetup{pdfstartview={XYZ null null 1}}
\usepackage{float}
\usepackage{bm}
\usepackage{tikz}
 %changes default sectioning commands -> 1,a, etc.
%\usepackage{breakurl}
%\renewcommand{\thesubsection}{(\alph{subsection})}
%\renewcommand{\thesubsubsection}{\roman{subsection}.}
\usepackage{lastpage}
\usepackage{fancyhdr}
\pagestyle{fancy}

%%% Header and Footer %%% 
\lhead{}
\chead{\leftmark}
\rhead{}
\lfoot{Gonzales and Delora }
\cfoot{Data Mining Proposal}
\rfoot{Page \thepage\ of \pageref{LastPage}}
\IfFileExists{upquote.sty}{\usepackage{upquote}}{}

\begin{document}

\title{CS591: Data Mining \\ Project Proposal \\ Mining Moods and Markets}
\author{Aaron Gonzales, Adam Delora}
\maketitle



\section{Introduction}


High level goal - Predict market behaviour correlation or relationship between market activity and semantic or sentiment analysis.

Twitter is a social networking service that allows users to send “tweets” that are 140 characters in length. Tweets can be used for things like personal communications, advertisements and public relations.  The stock market is a set of buyers and sellers that can buy or sell security. The purpose of the stock market is to raise money for a company. It allows for the buying and selling of ownership in a company. This also allows for the liquidity or the ability to quickly buy and sell ownership in a company. The stock market is often viewed in relationship to a country’s or company’s economic status and strength. Our goal is to see how the sentiment of tweets relates to the market.


\section{Data}

We will need two data sources for this project: 
\begin{itemize}
	\item Finance data
	\item linguistically rich data
\end{itemize}

\subsection{Finance Data}
\subsubsection{Sources}
Financial data is relatively easy to get - we plan on using the Yahoo finance
API for this purpose. The max history available through their api for hourly data is 5 days.
During the collection period, we will pull the data every to have
a complete history during our experimental period. We will collect hourly
updates for the period and the data is already in a clean CSV format, which
will require trivial organization on our part, though items will have to be
checked for date compatibility. 

\subsubsection{Stocks and Indices}
We will collect data on several popular, high-volume stocks:

\begin{itemize}
	\item Apple (AAPL)
	\item Google (GOOG)
	\item Microsoft (MSFT)
	\item Twitter (TWTR)
	\item Amazon (AMZN)
	\item Facebook (FB)
\end{itemize}

and we will also the NASDAQ 100 equity index for further analysis.


\subsection{Semantic Data}
Twitter is a social networking service that allows any user to post 140
character updates (``tweets'') to the site. Founded in 2006, it has grown to be
a top-ten visited website (cite?) and has 255 million monthly active users
producing approximately 500 million tweets per day. It has been a bountiful
data source for innumerable researchers across a motley assortment of
backgrounds, e.g., computer science, economics, sociologists, political
scientist, and linguistics.

\url{https://investor.twitterinc.com/releasedetail.cfm?ReleaseID=843245}
\url{http://www.sec.gov/Archives/edgar/data/1418091/000119312513390321/d564001ds1.htm#toc564001_1}

Twitter has two APIs available, one for Streaming data that provides direct access to the main stream of tweets


\begin{lstlisting}
{
  "coordinates": null,
  "created_at": "Thu Oct 21 16:02:46 +0000 2010",
  "favorited": false,
  "truncated": false,
  "id_str": "28039652140",
  "entities": {
    "urls": [
      {
        "expanded_url": null,
        "url": "http://gnip.com/success_stories",
        "indices": [
          69,
          100
        ]
      }
    ],
    "hashtags": [
 
    ],
    "user_mentions": [
      {
        "name": "Gnip, Inc.",
        "id_str": "16958875",
        "id": 16958875,
        "indices": [
          25,
          30
        ],
        "screen_name": "gnip"
      }
    ]
  },
  "in_reply_to_user_id_str": null,
  "text": "what we've been up to at @gnip -- delivering data to happy customers http://gnip.com/success_stories",
  "contributors": null,
  "id": 28039652140,
  "retweet_count": null,
  "in_reply_to_status_id_str": null,
  "geo": null,
  "retweeted": false,
  "in_reply_to_user_id": null,
  "user": {
    "profile_sidebar_border_color": "C0DEED",
    "name": "Gnip, Inc.",
    "profile_sidebar_fill_color": "DDEEF6",
    "profile_background_tile": false,
    "profile_image_url": "http://a3.twimg.com/profile_images/62803643/icon_normal.png",
    "location": "Boulder, CO",
    "created_at": "Fri Oct 24 23:22:09 +0000 2008",
    "id_str": "16958875",
    "follow_request_sent": false,
    "profile_link_color": "0084B4",
    "favourites_count": 1,
    "url": "http://blog.gnip.com",
    "contributors_enabled": false,
    "utc_offset": -25200,
    "id": 16958875,
    "profile_use_background_image": true,
    "listed_count": 23,
    "protected": false,
    "lang": "en",
    "profile_text_color": "333333",
    "followers_count": 260,
    "time_zone": "Mountain Time (US & Canada)",
    "verified": false,
    "geo_enabled": true,
    "profile_background_color": "C0DEED",
    "notifications": false,
    "description": "Gnip makes it really easy for you to collect social data for your business.",
    "friends_count": 71,
    "profile_background_image_url": "http://s.twimg.com/a/1287010001/images/themes/theme1/bg.png",
    "statuses_count": 302,
    "screen_name": "gnip",
    "following": false,
    "show_all_inline_media": false
  },
  "in_reply_to_screen_name": null,
  "source": "web",
  "place": null,
  "in_reply_to_status_id": null
}
\end{lstlisting}



\section{hypothesis}

There is a relationship between twitter sentiment and the movements of the stock market. 

\section{method}

\subsection{Data Capture and Storage}

Luckily, there are many tools available to help expedite our collection and pre-processing of the data. Tweets will be obtained during market hours (0900 - 1630 M-F, EST) using the Twitter streaming api, accessed via Python using the Tweepy package. Only the following information from tweets will be stored:

\begin{enumerate}
\item tweet text
\item created at
\item in reply to user
\item user id
\item location - author’s location
\item favorites count
\item statuses count
\item friends count
\item time zone
\item utc offset
\item lang
\item followers count
\item coordinates (geojson)
\item source
\end{enumerate}

As to reduce our data-processing and storage overhead. While network and location data are not focal points of our project, they will be used to help identify clusters of influential people if our hypothesis pans out. Tests estimate anywhere between 1.2-2.2 million tweets per day during our collection window, for a very rough estimate of 30 million total tweets collected. Initially, tweets will be saved as compressed JSON files. 


The stock market data will be taken from stooq. Stooq is a Polish website that offers free historical data in metastock and ASCII format for daily, hourly and 5 minute timeframes. The data consists of:
\begin{enumerate}
\item open
\item high
\item low
\item close
\item volume
\end{enumerate}
The storage overhead for this data will be low. The stock market runs from 9:30 am to 4:00 pm eastern time, which is 6 hours and 30 minutes. For the hourly data that is 7 timepoints per day and for the 5 minute data that is 78 timepoints. The 5 minute data is only stored for a week, so if we want to bin the data for different timeframes, we will have to collect this data every week.  The data will be initially stored in csv files.


Data will be stored together in a Mongo NoSQL database, chosen for its flexibility scalability, schemaless storage (each bin of financial data will have all associated tweets with that bin’s ‘json document’), and elegant queries. Working with the Mongo database will be handled through Python. 

We plan on collecting data over a one-month period starting 2014-10-13 and ending 2014-11-07. 

\subsection{Analysis}

\subsubsection{Cleaning and Organization}
After our data collection period, the tweets will be cleaned for consistency. We will discard tweets (possibly during collection) that are not from the United States or in English. If possible, Tweets will be filtered through a spam-detection tool to make the data less noisy. 

Data will be binned into one-hour bins and tweets within that timeframe will be associated with that bin. 

\subsubsection{Sentiment Analysis}
The AFINN database is a tagged lexical database used to quantify sentiment. There are 2477 words with semantic scores that can be used to get an aggregate sentiment value for a given tweet. The tweets during a specific timeframe will have an average semantic score for each time bin. Other sentiment measures may be used, time allowing. 

\subsubsection{Analysis}
We will use the time-series analysis and dimensionality reduction features within MATLAB to do economic modeling and a k-means clustering of our data. As our analysis is exploratory, initial modelling efforts will likely lead us to alternative analysis approaches, as will further ideas gained from this class. 


\section{Conclusion}
While other attempts have been made to predict market flow based on Twitter data, most have focused on a single equity index and have been limited to binary or categorical representations of mood. Our attempt is more broad in nature and can yield interesting, if any, information on relationships between people’s words and the market. 


\section{References}

\end{document}
